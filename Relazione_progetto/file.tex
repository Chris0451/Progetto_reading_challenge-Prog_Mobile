\documentclass{article}
\usepackage{graphicx, float}
\usepackage[hidelinks]{hyperref}

\graphicspath{{images/}}
\author{Cristian Di Cintio - S1110150}

\title{RELAZIONE PROGETTO - PROGRAMMAZIONE MOBILE}


\begin{document}

\maketitle
\tableofcontents
\newpage

\section{Introduzione}

L'applicazione da sviluppare permette all'utente di registrare o ricercare i libri che vuole leggere in un proprio catalogo personale con la possibilità di effettuare sfide di conoscenza, sotto forma di quiz, proposte tra utenti sui libri terminati.

Inoltre ogni utente può creare una lista personalizzata di libri scelti per la lettura (es. Libri letti, Libri Fantasy preferiti ecc.).

L'utente può anche decidere degli orari nei quali leggere durante il giorno, con l'arrivo di notifiche personalizzate e la possibilità di fare uso di un timer per la lettura. Tale timer permette anche, durante la sua funzione, di non far ricevere norifiche indesiderate durante la lettura. Al termine della giornata, ci sarà un resoconto, aggiornato dall'utente, del numero di pagine letto di un determinato libro.

Tale resoconto viene salvato in uno storico personale per la visualizzazione dei propri progressi di lettura durante le sessioni.

Si permette una migliore stimolazione della lettura e la possibilità di interagire tra utenti, con la presenza di una lista di amici, con i propri titoli preferiti.

Gli utenti possono interagire tra di loro anche attraverso commenti, recensioni sui singoli titoli e anche tramite dei "Mi piace" verso un determinato libro, per accrescerne una certa popolarità

La lista di libri viene inizialmente fornita tramite delle API di Google ma gli utenti possono anche inserire un libro non presente (ad esempio libri nuovi) in modo da tale da avere un catalogo aggiornato grazie ai lettori stessi.

L'applicazione fornisce un metodo di ricerca di libri per titolo, autore e codice ISBN da poter scannerizzare tramite la fotocamera del cellulare, nel caso si abbia un libro fisico.

Lo sviluppo viene effettuato con Kotlin tramite l'ambiente di sviluppo Android Studio, in modo tale da fornire un'interfaccia semplice, intuitiva e funzionale.

Per ampliare l'accessibilità anche per gli utenti IOs, si è deciso di realizzare tale applicazione anche per gli utenti Apple tramite lo sviluppoin Flutter.

La relazione ha come obiettivo l'illustrazione dei passi da effettuare per la sua realizzazione, includendo le scelte effettuate e i motivi per cui sono state implementate determinate funzionalità.

\section{Progettazione}

\subsection{Glossario dei termini}
\subsection{Requisiti}
\subsubsection{Requisiti funzionali}
\subsubsection{Requisiti non funzionali}

\subsection{Casi d'uso}

\subsection{Diagramma dei componenti}



\section{Programmazione in Android}

\section{Progrmmazione in Flutter}

\section{Errori e possibili bug}

\end{document}
